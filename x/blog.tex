\documentclass[a4paper, oneside]{memoir}
\usepackage{fontspec}
\usepackage{unicode-math}
\defaultfontfeatures{Ligatures=TeX}
\setmainfont[
  BoldFont=texgyrepagella-bold.otf,
  ItalicFont=texgyrepagella-italic.otf,
  BoldItalicFont=texgyrepagella-bolditalic.otf]
  {texgyrepagella-regular.otf}
\setmathfont{texgyrepagella-math.otf}
\linespread{1.05} %% Palatino needs a little more leading
\chapterstyle{bringhurst}

\begin{document}

\chapter*{100}

After 9 years of blogging on other peoples platforms, it is due time I start writing documents in my own directories. Since I don't like HTML much, the way I write documents is by using \LaTeX{}. I may be exporting the documents somewhere else, but I have a deep love for old-style typography.

We start off at 100 because it means I can go back and add the backlog of stuff to the file when needed in the future and when I decide to move stuff fully to this new format. I have been somewhat prolific, but I have not reached 100 entries yet, luckily.

Why use an old ``archaic'' format? First, it prints well. Second, it truly removes the burden of the writer to do typography. Third, it allows proper mathematical symbolism to be used. Want a $\mathcal{F}\; $? You have it. Need to pick a natural number? $n \in \mathbb{N}$ will do! You don't get the advantage of such notation in most blogging platforms. Furthermore, most of my writing is technical in nature. I often include snippets of code, and in \LaTeX{} this is also fairly easy to include. Whereas on many blogging platforms, I spend time formatting stuff. While I still have to spend some of that time when converting the documents from here to the format, not having the formatting overhead when initially writing stuff is nice.

\end{document}